\documentclass{article}
\usepackage{amsmath}
\usepackage{hyperref}
\usepackage{xcolor}
\usepackage{ulem}

\definecolor{darkblue}{rgb}{0, 0, 20}

\hypersetup{
    colorlinks=true,
    urlcolor=darkblue,
    linkcolor=blue,
    filecolor=magenta,
    citecolor=blue,
}

\title{Real-Time Dynamic Robot-Assisted Hand-Object Interaction via Motion Primitives}
\author{Mingqi Yuan, Huijiang Wang, Kai-Fung Chu, \\ Fumiya Iida, Bo Li, Wenjun Zeng}
\date{}
\setlength{\parindent}{0pt}

\begin{document}

\maketitle

\begin{center}\textbf{\href{https://arxiv.org/pdf/2405.19531}{Paper}}\end{center}

This paper proposes a new method of dynamic robot-assisted hand-object interaction using hand pose estimation, adaptive robot control, and motion primitives. This method includes a transformer-based algorithm to 3D model the human hands from single-view RGB images, with a motion primitives model to translate human hand motions into robotic actions.

\section*{Motivation}

There are current challenges in physical human-robot interaction, such as accurate real-time human motion perception, adaptive robot control, and effective human-robot. This paper introduces a system to improve robot assistance in tasks that require fine-grained physical collaboration.

\section*{Architecture}

\subsubsection*{3D Hand Pose Sensing}

The system first estimates the 3D hand pose from a single RGB image using a transformer-based model, MeshGraphormer. A moving average filter smooths oscillations in hand pose data caused by detection noise.

\subsubsection*{Motion Primitives Model (MPM)}

A bidirectional LSTM processes time-series data of 3D hand joint coordinates to retain temporal context, and then map dynamic hand motions to predefined robotic actions, such as moving forward or grasping.

\subsubsection*{Robot Controller}

The system lastly integrates open-loop control for both predefined actions with closed-loop control for dynamic, adaptive responses. Open-loop control executes predefined motions when motion primitives are detected, while closed-loop control continuously adjusts the tobot's tool center point (TCP) based on real-time feedback.

\section*{Limitations}

\begin{enumerate}
    \item The system is trained on predefined motions and specific tasks.
    \item The system has not been tested with various environments or complex objects.
\end{enumerate}

\end{document}