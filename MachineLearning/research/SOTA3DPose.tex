% chktex-file 13

\documentclass{article}
\usepackage{amsmath}
\usepackage{hyperref}
\usepackage{xcolor}
\usepackage{ulem}
\usepackage{graphicx}
\usepackage[margin=0.75in]{geometry}

\graphicspath{ {./images/} }

\definecolor{darkblue}{rgb}{0, 0, 20}

\hypersetup{
    colorlinks=true,
    urlcolor=darkblue,
    linkcolor=blue,
    filecolor=magenta,
    citecolor=blue,
}

\setlength{\parindent}{0pt}

\title{SOTA Methods in 3D Hand and Body Pose from CVPR 2024 and ECCV 2024}
\author{}
\date{}

\begin{document}

\maketitle

\section*{HandDAGT}

\begin{itemize}
    \item Conference: ECCV 2024
    \item Authors: Wencan Cheng, Eunji Kim, Jong Hwan Ko
    \item Hand or Body: Hand
    \item Motivation: Existing graph methods in 3D hand pose estimation use static graphs which are unable to capture dynamic kinematic relations between joints in occluded scenarios. Also, self-occlusion and hand-object occlusion are major challenges for current methods.
    \item Method: The 3D hand point cloud and depth map are used to generate keypoint embeddings and determine a 3d point patch for each joint. The embeddings and patches are passed as queries and keys to an adaptive graph transformer, which dynamically balances local attention for visible keypoints and kinematic attention for occluded ones.
    \item Limitations: The model can't process interacting hands and focuses on single-hand samples. Also, this method requires redundant computation due to the 2D feature extraction. They state that possible solutions include bidirectional learning and developing a lightweight 2D model.
    \item Paper: https://arxiv.org/pdf/2407.20542
    \item GitHub: https://github.com/cwc1260/HandDAGT
\end{itemize}

\section*{HandDiff}

\begin{itemize}
    \item Conference: CVPR 2024
    \item Authors: Wencan Cheng, Hao Tang, Luc Van Gool, Jong Hwan Ko
    \item Hand or Body: Hand
    \item Motivation: Depth image-based methods either use 2D CNN-based approaches, which do not accurately capture the 3D structure, or 3D CNN-based approaches, which require large amounts of memory and computation. More recently, PointNet-based methods that process the 3D point cloud use a sparse point cloud to reduce computation at the cost of performance. Previous diffusion-based models rely on global features and overlook local details. Also, they are permutation-equivariant, which limits their ability to distinguish between joints.
    \item Method: The depth map and 3D point cloud are processed separately to generate 2D and 3D features. They are then concatenated and passed through three layers of BIL to generate joint-wise embeddings. The denoiser starts with a randomly sampled set of joint coordinates from a Gaussian distribution, and then for each time step out of a set amount, the denoiser samples local features around each joint and refines the joint coordinates using a kinematic-aware GCN.
    \item Limitations: The model can't process interacting hands and focuses on single-hand samples. Also, the denoiser's performance plateaus after around 5 timesteps.
    \item Paper: https://arxiv.org/pdf/2404.03159
    \item GitHub: https://github.com/cwc1260/HandDiff
\end{itemize}

\section*{HOISDF}

\begin{itemize}
    \item Conference: CVPR 2024
    \item Authors: Haozhe Qi, Chen Zhao, Mathieu Salzmann, Alexander Mathis
    \item Hand or Body: Hand
    \item Motivation: Direct lifting methods do not utilize 3D intermediate representations and rely on the network to learn the mapping from 2D image to 3D pose. Coarse-to-fine methods make initial predictions and then refine them with explicit intermediate representations, such as hand joints or vertices, but require more computation and is not as precise as implicit intermediate representations, such as a signed distance field as a continuous function. Also, current signed distance field applications mainly use them to directly reconstruct 3D meshes instead of as an intermediate representation.
    \item Method: This method estimates signed distance field for both the hand and object to build the shape of each. The 3D space is split into many bins, each with a query point, and the ones on the surface, which have the lowest distance to the surface, are then used to extract hand and object features. These features are enhanced with multi head self-attention layers and then used to compute the 3D hand joints and mesh, as well as the object rotation and translation vectors. Each joint's 3D position is fine-tuned with the distance vectors from each nearby query point.
    \item Limitations: The hand and object meshes might intersect with each other in severely occluded scenarios. Also, this model was only trained on objects that are bigger than the hand, which are much less occluded than smaller objects.
    \item Paper: https://arxiv.org/pdf/2402.17062
    \item GitHub: https://github.com/amathislab/HOISDF
\end{itemize}

\section*{WildHands}

\begin{itemize}
    \item Conference: ECCV 2024
    \item Authors: Aditya Prakash, Ruisen Tu, Matthew Chang, Saurabh Gupta
    \item Hand or Body: Hand
    \item Motivation: No existing method performs well in egocentric views as the perspective is distorted and there is a lack of 3D annotatins on the wild.
    \item Method: The model starts with cropped hand images and camera intrinsic parameters and mitigates the perspective distortion by encoding the hand's position in the camera's field of view with sinusoidal functions. The encodings are then passed through convolutional layers and a 1-layer MLP to generate a feature vector. The MANO parameters are initialized at 0 and then each of the parameters are passed through a 3-layer MLP with the feature vector and then added back to the parameters, three times. The result is the final MANO prediction.
    \item Limitations: The model does not perform well on images where the fingers are barely visible or contain complex poses. The KPE encoding requires camera intrinsic parameters. Also, they manually set hyperparameters for loss term weights.
    \item Paper: https://arxiv.org/pdf/2312.06583
    \item GitHub: https://github.com/ap229997/hands
\end{itemize}

\section*{NC-RetNet}

\begin{itemize}
    \item Conference: ECCV 2024
    \item Authors: Kaili Zheng, Feixiang Lu, Yihao Lv, Liangjun Zhang, Chenyi Guo, Ji Wu
    \item Hand or Body: Body
    \item Motivation: Most previous methods lift 2D keypoints to 3D but it is not very accurate as one 2D detection may correspond to multiple 3D skeletons. Previous methods that process videos and extract temporal information, process the frames with a sliding-window that treats past and future frames equally, but a larger chunk size means relying on more future frames before inference, which increases inference latency.
    \item Method: This model is built off the RetNet architecture but with non-causal masking to incorporate future frames. They implement a chunkwise recurrent representation of RetNet, where the video sequence is split into chunks and the chunks are processed in parallel but the frames in the chunk are processed recurrently. This includes a cross-chunk state that tracks long-term features. Also, Rotary Position Encoding (RPE), a type of relative positional encoding, handles temporal relationships between frames in the attention mechanism of RetNet. They train on larger chunks to capture more temporal relationships but then transfer knowledge from large chunks to smaller chunks at inference to reduce latency.
    \item Limitations: The theory behind the method's ability to trasnfer knowledge from large chunks to small chunks is unclear. Also, the method has only been tested for 2D-to-3D lifting.
    \item Paper: https://www.ecva.net/papers/eccv\_2024/papers\_ECCV/papers/04820.pdf
    \item GitHub: https://github.com/Kelly510/PoseRetNet
\end{itemize}

\section*{PAFUSE}

\begin{itemize}
    \item Conference: ECCV 2024
    \item Authors: Nermin Samet, Cédric Rommel, David Picard, Eduardo Valle
    \item Hand or Body: Hand and Body
    \item Motivation:
    \item Method:
    \item Limitations:
    \item Paper: https://arxiv.org/pdf/2407.10220
    \item GitHub: https://github.com/valeoai/PAFUSE
\end{itemize}

\end{document}

% \begin{itemize}
%     \item Conference:
%     \item Authors:
%     \item Hand or Body:
%     \item Motivation:
%     \item Method:
%     \item Limitations:
%     \item Paper: 
%     \item GitHub:
% \end{itemize}