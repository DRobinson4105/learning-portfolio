\documentclass{article}
\usepackage{amsmath}
\usepackage{relsize}
\usepackage[a4paper, top=0.75in, bottom=0.75in]{geometry}

\setlength{\parindent}{0pt}

\begin{document}

Targets
\begin{enumerate}
    \item Range of motion
    \item Time to complete event (pick up a block, move it, place it down)
    \item Which fingers are being used to pick the block
    \item Smoothness of motion
\end{enumerate}

\section*{Determine Hand-Object Interaction}

\begin{enumerate}
    \item Box-block test uses unit cubes so vertices are shifted by $0.5$.
    \[b=\begin{bmatrix} 0.5 \\ 0.5 \\ 0.5 \end{bmatrix}\]
    \item Transforms the query point into the cube's local space
    \[p_\text{local}=R^{-1}(p-t)\]
    \item Create SDF for block
    \[SDF_\text{block}(p)={\|\max(p_\text{local})\|}_2 + \min(\max(p_\text{local}), 0)\]
    \item Filter keypoints for fingertips
    \begin{center}\verb|thumb = kpts[4], fingers = [kpts[8], kpts[12], kpts[16], kpts[20]]|\end{center}
    \item Function to determine if keypoint is in contact with block
    \[\text{in\_contact}(x)=\text{SDF}_\text{block}(x) < \mathlarger{\epsilon}\quad\text{where }\mathlarger{\epsilon}=\text{distance threshold}\]
    \item Hand is holding object if
    \begin{center}\verb|in_contact(thumb) and any(in_contact(kpt) for kpt in fingers)|\end{center}
\end{enumerate}

\section*{Range of motion}

\begin{enumerate}
    \item Concatenate 2D body keypoints with depth values to form 3D body keypoints
    \item Track shoulder and elbow angles
    \[\theta(A, B, C) = \arccos \Bigg(\frac{(A-B)\cdot (C-B)}{\| A-B \|\| C-B \|}\Bigg)\]
    \begin{verbatim}
left_shoulder = kpts[8], kpts[6], kpts[12]
right_shoulder = kpts[9], kpts[7], kpts[13]
left_elbow = kpts[6], kpts[8], kpts[10]
right_elbow = kpts[7], kpts[9], kpts[11]
    \end{verbatim}
\end{enumerate}

\section*{Time to complete event}
Track last event and HOI state.
\vspace{1em}
\begin{verbatim}
if hand is holding object:
    if hand wasn't holding object before:
        object has been picked up
    else if hand has moved moved past a certain threshold (to the other side):
        object has been moved
else:
    if hand was holding object:
        object has been dropped
\end{verbatim}

\section*{Which fingers were used to pick up block}

Can be checked when determining hand-object interaction

\section*{Smoothness of motion}

\begin{enumerate}
    \item Compute velocity
    \begin{center}\verb|velocities = diff(keypoints)| $\leftarrow v_t=k_{t+1}-k_t$\end{center}
    \item Compute acceleration
    \begin{center}\verb|accelerations = diff(velocities)| $\leftarrow a_t=v_{t+1}-v_t$\end{center}
    \item Compute jerk
    \begin{center}\verb|jerk = diff(accelerations)| $\leftarrow j_t=a_{t+1}-a_t$\end{center}
    \item Lower RMS means more smooth motion
    \begin{center}\verb|rms = sqrt(mean(jerk ** 2))|\end{center}
    \[j_\text{rms}=\sqrt{\frac{1}{n}\sum_{t=1}^{n}j_t^2}\]
\end{enumerate}
\end{document}